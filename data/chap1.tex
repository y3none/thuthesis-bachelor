% !TeX root = ../thuthesis-yyy.tex

\chapter{引言}


\section{课题背景}

(课题背景与目标。)

操作系统(Operating System,OS)是计算机系统的核心软件,承担着管理和协调计算机硬件与软件资源的重任,其性能和功能直接影响着整个系统的运行效率和用户体验。而以Linux为代表的宏内核操作系统凭借其在资源管理和系统调用方面的高效性,在服务器、个人计算机等领域被广泛应用,支撑着大量业务的稳定运行与高效管理。但随着信息技术领域的快速发展,云计算、物联网等新兴技术不断涌现,计算机设备的应用场景日益复杂,对操作系统的灵活性和可扩展性提出了更高的要求。面对种类繁多、功能各异的设备,操作系统需要灵活适配、快速响应不同设备的多样化需求,而传统的宏内核操作系统将所有功能均放在单一的内核空间中实现,服务高度集中化,虽然能避开复杂的进程间通信机制,减少用户态和内核态之间的切换开销,提高系统调用效率,却也导致了内核代码规模庞大,可维护性和可扩展性较差,难以复用到资源差异较大的异构平台,而未熟练掌握内核开发技能的用户又难以针对不同场景需求开发适配的内核,因此一个能支持用户自由搭建所需功能的组件化宏内核操作系统便显得尤为重要(BrickOS: 面向异构硬件资源的积木式内核)。

任务管理作为宏内核操作系统的核心功能,负责高效调度任务、合理分配资源等关键工作,从而保障整个系统的有序运行,因此在搭建组件化操作系统的过程中首先且必须要实现的便是任务管理组件。(补充)

本课题旨在基于支持组件可重用设计的组件化操作系统基层架构ArceOS,搭建其上层系统调用,从而设计并封装出能直接支持Linux应用的组件化宏内核操作系统的任务管理组件,帮助定制更便捷、高效、安全的宏内核操作系统。



\section{相关研究工作}

(
1. 组件化操作系统的理论研究现状;

2. 对已有的OS实现现状(如byteos、dragonos、新绽)进行简单分析与介绍。
)

在组件化操作系统的理论研究方面,众多学者均对其架构方式、设计原则以及运行机制进行了深入的探讨。(待补充 综述)

目前,已有多个操作系统对宏内核操作系统进行了各自的设计尝试与实现,其中ByteOS、DragonOS以及星绽操作系统均使用Rust语言开发并提供了完备的宏内核相关功能,兼容Linux应用,他们的实现为本课题的设计研究提供了宝贵的借鉴经验。

ByteOS是一个开源的操作系统项目,将进程管理、内存管理、文件系统等核心模块紧密集成,实现了较为简洁的宏内核架构。在任务管理方面,ByteOS使用进程控制块(PCB)作为管理进程的数据结构,记录进程状态、优先级、程序计数器等关键信息,线程管理同样使用类似的线程控制块(TCB)来管理线程的上下文信息,包括陷阱帧、信号掩码、子线程退出标志、信号队列、退出信号和线程退出码等。ByteOS支持多种硬件平台,如riscv64、aarch64、x86\_64、loongarch64等,并能在这些平台上运行Linux应用程序。

DragonOS同样基于宏内核架构,是一个面向云计算轻量化场景的,完全自主内核的64位操作系统。其内核包含完备的宏内核操作系统所需功能,并实现了优秀的内存管理模块,对内核空间和用户空间的内存分配、释放、管理等进行了封装,在任务管理方面也实现了完善的进程管理和多核调度,并支持内核线程和kthread机制,为内核线程的创建和管理提供了方便的接口,保证系统能高效处理各种任务。

星绽操作系统则致力于充分发挥Rust潜力, 以安全高效的方式实现Linux ABI并满足专有内核模块的业务需求。星绽操作系统的任务管理机制在任务创建、调度、同步与通信以及退出等方面都有其独特的设计和实现,在任务调度上也会根据任务的状态和资源需求进行动态调整。



\section{组件化操作系统}

(
1. 结合上面的研究现状,介绍什么是宏内核下的组件化操作系统;

2. 介绍任务管理模块:进程与线程、任务状态与生命周期、任务调度策略;

3. 简要描述宏内核架构下的任务管理地位,并说明任务管理是怎样参与到一个操作系统宏内核的构建与运行中去的。
)

宏内核下的组件化操作系统在传统宏内核基础架构上进行创新发展,将操作系统的各个核心功能封装为独立的组件,各功能不再紧耦合,而是在通过组件间的标准化接口进行交互与协作。这种架构模式既能保留宏内核的高性能优势,又能赋予系统良好的可维护性与灵活的可扩展性,从而大幅拓展宏内核操作系统的应用范围。

任务管理模块作为组件化操作系统的关键组成部分,通过管理进程与线程、控制任务状态与生命周期以及灵活进行任务调度来实现对宏内核核心功能的支持。进程作为系统资源分配和CPU调度的基本单位,是程序在计算机中的一次动态执行过程,每个进程都拥有独立的地址空间及系统资源,一般通过进程控制块来实现对进程状态、优先级、资源占用等信息的精确记录。而线程作为进程内的轻量级执行单元,则通过线程控制块来实现,通过共享进程的资源实现协同工作,但也拥有独立的线程栈和程序计数器以实现独立调度,保证资源的高效利用。任务状态描述了任务在执行过程中的不同阶段,常见的任务状态包括就绪态、运行态、阻塞态和终止态,任务状态的转换构成了任务的生命周期,通过任务管理模块进行监控和控制。而任务调度策略的核心目标是平衡公平性与效率,基于此存在一系列策略对任务的状态进行管理与切换,常见策略包括时间片轮转(Round-Robin,RR)、先来先服务(First Come First Serve,FCFS)、多级反馈队列调度(Multi-Level Feedback Queue, MLFQ)等,通过灵活运用各种调度策略,操作系统能根据任务特点和资源情况对CPU资源进行高效分配,提升系统效率,满足不同应用场景的需求。

在宏内核架构中,任务管理模块贯穿操作系统的整个运行过程,在系统启动阶段负责任务相关数据结构与资源的初始化,在系统运行过程中则与内存管理、文件系统等模块协作,调度任务、管理资源,并在系统运行结束后回收相关资源,保障系统能够高效、稳定地运行。


\section{本文工作}

(概述本文主要工作,如何基于单位内核组件实现直接支持Linux应用的组件化宏内核。)

本文旨在实现高效、完备、可靠的宏内核任务管理组件,将基于基座代码ArceOS中的单位内核组件以及Starry-Next整体框架展开深入研究与实践,通过对任务管理相关数据结构的设计运用、系统调用接口的补全实现以及对整个任务管理组件功能与性能的集成测试,实现对组件化宏内核的核心支持。

此外,本课题同时为ArceOS及Starry-Next框架中的任务相关功能与接口提供了详细的说明文档,以供使用者参阅。

本文分为6个章节,在第一章介绍课题背景及组件化宏内核相关研究工作,在第二章将介绍本课题所使用的基座代码ArceOS的整体架构以及任务管理相关单元模块,在第三章则将对Starry-Next框架进行介绍、接口分析及任务管理模块的相关功能,在第四章将介绍任务管理组件的设计与实现,在第五、六章则将分别对组件进行测试分析以及最终总结。