% !TeX root = ../thuthesis-yyy.tex

\chapter{引言}


\section{课题背景}

课题背景与目标。



\section{相关研究工作}

1. 组件化操作系统的理论研究现状;

2. 对已有的OS实现现状(如byteos、dragonos)进行简单分析与介绍。



\section{组件化操作系统}

1. 结合上面的研究现状,介绍什么是宏内核下的组件化操作系统;

2. 介绍任务管理模块:进程与线程、任务状态与生命周期、任务调度策略;

3. 简要描述宏内核架构下的任务管理地位,并说明任务管理是怎样参与到一个操作系统宏内核的构建与运行中去的。
% 论文摘要是对论文研究内容的高度概括,应具有独立性和自含性,即应是 一篇简短但意义完整的文章。
% 通过阅读论文摘要,读者应该能够对论文的研究 方法及结论有一个整体性的了解,因此摘要的写法应力求精确简明。
% 论文摘要 应包括对问题及研究目的的描述、对使用的方法和研究过程进行的简要介绍、 对研究结论的高度凝练等,重点是结果和结论。

% 论文摘要切忌写成全文的提纲,尤其要避免“第 1 章……;第 2 章……;……”这样的陈述方式。



\section{本文工作}

概述本文主要工作,如何基于单位内核组件实现直接支持Linux应用的组件化宏内核。