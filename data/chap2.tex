% !TeX root = ../thuthesis-yyy.tex

\chapter{基座代码架构分析}

\section{整体架构概述}

(介绍ArceOS,给出总体分析。)

ArceOS是一个使用Rust语言编写的实验性模块化微虚拟机操作系统,它具有高度的可定制性和灵活性,旨在提供一个高效、安全且易于扩展的操作系统平台,为各种特定场景的应用提供可定制性的操作系统底层支撑。ArceOS采用模块化设计,以比传统设计更细粒度的划分将操作系统拆分为多个独立的模块,如运行时模块、任务管理模块、设备驱动模块等,并依赖Rust语言的包管理机制来实现模块的组件化,为每个模块提供明确的功能和接口,模块之间通过接口进行良好交互。ArceOS中的组件分为两类,与操作系统功能无关的称为元件,位于最底层,为其上层的组件提供可重用的数据结构、硬件处理等相关操作,而与操作系统相关的组件则称为模块,位于需要依赖特定操作系统的需求进行特定的修改,其位于元件的上层,通过调用元件层的基础功能实现对操作系统基本功能的支持。

由于ArceOS的底层元件具有高度的可定制性和灵活性,其可以针对不同的应用需求定制出轻量级的操作系统,提高资源利用率和性能,这也为本课题基于ArceOS开发组件化宏内核操作系统提供了便利的支撑,在实现过程中不再需要深入探究可复用的底层操作系统知识,而只需关注相关接口以及模块层提供的功能支持,针对宏内核的功能需求对模块层进行相关调用与修改,将其封装成完备的系统调用与宏内核操作系统进行交互,即可便利快捷地实现操作系统上层的任务管理组件,实现任务的创建、调度等管理需求。


\section{基座代码的任务管理组件}

(
对 ArceOS 任务管理相关核心组件进行介绍,功能介绍与实现分析:

1. axruntime:运行时;

2. axtask:任务管理与调度;

3. axns:命名空间。
)

axruntime是ArceOS的运行时库,任何使用ArceOS的应用程序都需要链接该库。它负责在进入应用程序的main函数之前进行一系列的初始化工作,确保系统在一个合适的状态下运行用户程序。




\section{实现基础宏内核操作系统}

(如何使用基座框架实现基础宏内核操作系统并运行简单用户程序,这一部分之前总结过,需将其整理拟合成符合论文要求的表达。)


