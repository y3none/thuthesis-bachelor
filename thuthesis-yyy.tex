% !TeX encoding = UTF-8
% !TeX program = xelatex
% !TeX spellcheck = en_US

\documentclass[degree=bachelor]{thuthesis}
  % 学位 degree:
  %   doctor | master | bachelor | postdoc
  % 学位类型 degree-type:
  %   academic(默认)| professional
  % 语言 language
  %   chinese(默认)| english
  % 字体库 fontset
  %   windows | mac | fandol | ubuntu
  % 建议终版使用 Windows 平台的字体编译


% 论文基本配置,加载宏包等全局配置
% !TeX root = ./thuthesis-example.tex

% 论文基本信息配置

\thusetup{
  %******************************
  % 注意:
  %   1. 配置里面不要出现空行
  %   2. 不需要的配置信息可以删除
  %   3. 建议先阅读文档中所有关于选项的说明
  %******************************
  %
  % 输出格式
  %   选择打印版(print)或用于提交的电子版(electronic),前者会插入空白页以便直接双面打印
  %
  output = print,
  % 格式类型
  %   默认为论文(thesis),也可以设置为开题报告(proposal)
  % thesis-type = proposal,
  %
  % 标题
  %   可使用“\\”命令手动控制换行
  %
  % title  = {清华大学学位论文 \LaTeX{} 模板\\使用示例文档 v\version},
  title  = {操作系统宏内核的任务管理组件设计与实现},
  title* = {An Introduction to \LaTeX{} Thesis Template of Tsinghua
            University v\version},
  %
  % 学科门类
  %   1. 学术型
  %      - 中文
  %        需注明所属的学科门类,例如:
  %        哲学、经济学、法学、教育学、文学、历史学、理学、工学、农学、医学、
  %        军事学、管理学、艺术学
  %      - 英文
  %        博士:Doctor of Philosophy
  %        硕士:
  %          哲学、文学、历史学、法学、教育学、艺术学门类,公共管理学科
  %          填写“Master of Arts“,其它填写“Master of Science”
  %   2. 专业型
  %      直接填写专业学位的名称,例如:
  %      教育博士、工程硕士等
  %      Doctor of Education, Master of Engineering
  %   3. 本科生不需要填写
  %
  degree-category  = {工学硕士},
  degree-category* = {Master of Science},
  %
  % 培养单位
  %   填写所属院系的全名
  %
  department = {计算机科学与技术系},
  %
  % 学科
  %   1. 研究生学术型学位,获得一级学科授权的学科填写一级学科名称,其他填写二级学科名称
  %   2. 本科生填写专业名称,第二学位论文需标注“(第二学位)”
  %
  discipline  = {计算机科学与技术},
  discipline* = {Computer Science and Technology},
  %
  % 专业领域
  %   1. 设置专业领域的专业学位类别,填写相应专业领域名称
  %   2. 2019 级及之前工程硕士学位论文,在 `engineering-field` 填写相应工程领域名称
  %   3. 其他专业学位类别的学位论文无需此信息
  %
  % professional-field  = {计算机技术},
  % professional-field* = {Computer Technology},
  %
  % 姓名
  %
  author  = {俞颖妍},
  author* = {Yu Yingyan},
  %
  % 学号
  % 仅当书写开题报告时需要(同时设置 `thesis-type = proposal')
  %
  % student-id = {2000310000},
  %
  % 指导教师
  %   中文姓名和职称之间以英文逗号“,”分开,下同
  %
  supervisor  = {戴桂兰, 助理研究员},
  supervisor* = {Research Associate Dai Guilan},
  %
  % 副指导教师
  %
  associate-supervisor  = {陈渝, 副教授},
  associate-supervisor* = {Associate Professor Chen Yu},
  %
  % 联合指导教师
  %
  % co-supervisor  = {某某某, 教授},
  % co-supervisor* = {Professor Mou Moumou},
  %
  % 日期
  %   使用 ISO 格式;默认为当前时间
  %
  % date = {2019-07-07},
  %
  % 是否在中文封面后的空白页生成书脊(默认 false)
  %
  include-spine = false,
  %
  % 密级和年限
  %   秘密, 机密, 绝密
  %
  % secret-level = {秘密},
  % secret-year  = {10},
  %
  % 博士后专有部分
  %
  % clc                = {分类号},
  % udc                = {UDC},
  % id                 = {编号},
  % discipline-level-1 = {计算机科学与技术},  % 流动站(一级学科)名称
  % discipline-level-2 = {系统结构},          % 专业(二级学科)名称
  % start-date         = {2011-07-01},        % 研究工作起始时间
}

% 载入所需的宏包

% 定理类环境宏包
\usepackage{amsthm}
% 也可以使用 ntheorem
% \usepackage[amsmath,thmmarks,hyperref]{ntheorem}

\thusetup{
  %
  % 数学字体
  % math-style = GB,  % GB | ISO | TeX
  math-font  = xits,  % stix | xits | libertinus
}

% 可以使用 nomencl 生成符号和缩略语说明
% \usepackage{nomencl}
% \makenomenclature

% 表格加脚注
\usepackage{threeparttable}

% 表格中支持跨行
\usepackage{multirow}

% 固定宽度的表格。
% \usepackage{tabularx}

% 跨页表格
\usepackage{longtable}

% 算法
\usepackage{algorithm}
\usepackage{algorithmic}

% 量和单位
\usepackage{siunitx}

% 参考文献使用 BibTeX + natbib 宏包
% 顺序编码制
\usepackage[sort]{natbib}
\bibliographystyle{thuthesis-numeric}

% 著者-出版年制
% \usepackage{natbib}
% \bibliographystyle{thuthesis-author-year}

% 生命科学学院要求使用 Cell 参考文献格式(2023 年以前使用 author-date 格式)
% \usepackage{natbib}
% \bibliographystyle{cell}

% 本科生参考文献的著录格式
% \usepackage[sort]{natbib}
% \bibliographystyle{thuthesis-bachelor}

% 参考文献使用 BibLaTeX 宏包
% \usepackage[style=thuthesis-numeric]{biblatex}
% \usepackage[style=thuthesis-author-year]{biblatex}
% \usepackage[style=gb7714-2015]{biblatex}
% \usepackage[style=apa]{biblatex}
% \usepackage[style=mla-new]{biblatex}
% 声明 BibLaTeX 的数据库
% \addbibresource{ref/refs.bib}

% 定义所有的图片文件在 figures 子目录下
\graphicspath{{figures/}}

% 数学命令
\makeatletter
\newcommand\dif{%  % 微分符号
  \mathop{}\!%
  \ifthu@math@style@TeX
    d%
  \else
    \mathrm{d}%
  \fi
}
\makeatother

% hyperref 宏包在最后调用
\usepackage{hyperref}



\begin{document}

% 封面
\maketitle

% 使用授权的说明
% 本科生开题报告不需要
\copyrightpage
% 将签字扫描后授权文件 scan-copyright.pdf 替换原始页面
% \copyrightpage[file=scan-copyright.pdf]

\frontmatter
% !TeX root = ../thuthesis-yyy.tex

% 中英文摘要和关键字

\begin{abstract}
  论文的摘要是对论文研究内容和成果的高度概括。
  摘要应对论文所研究的问题及其研究目的进行描述,对研究方法和过程进行简单介绍,对研究成果和所得结论进行概括。
  摘要应具有独立性和自明性,其内容应包含与论文全文同等量的主要信息。
  使读者即使不阅读全文,通过摘要就能了解论文的总体内容和主要成果。

  论文摘要的书写应力求精确、简明。
  切忌写成对论文书写内容进行提要的形式,尤其要避免“第 1 章……;第 2 章……;……”这种或类似的陈述方式。

  关键词是为了文献标引工作、用以表示全文主要内容信息的单词或术语。
  关键词不超过 5 个,每个关键词中间用分号分隔。

  % 关键词用“英文逗号”分隔,输出时会自动处理为正确的分隔符
  \thusetup{
    keywords = {关键词 1, 关键词 2, 关键词 3, 关键词 4, 关键词 5},
  }
\end{abstract}

\begin{abstract*}
  An abstract of a dissertation is a summary and extraction of research work and contributions.
  Included in an abstract should be description of research topic and research objective, brief introduction to methodology and research process, and summary of conclusion and contributions of the research.
  An abstract should be characterized by independence and clarity and carry identical information with the dissertation.
  It should be such that the general idea and major contributions of the dissertation are conveyed without reading the dissertation.

  An abstract should be concise and to the point.
  It is a misunderstanding to make an abstract an outline of the dissertation and words “the first chapter”, “the second chapter” and the like should be avoided in the abstract.

  Keywords are terms used in a dissertation for indexing, reflecting core information of the dissertation.
  An abstract may contain a maximum of 5 keywords, with semi-colons used in between to separate one another.

  % Use comma as separator when inputting
  \thusetup{
    keywords* = {keyword 1, keyword 2, keyword 3, keyword 4, keyword 5},
  }
\end{abstract*}


% 目录
\tableofcontents

% % 插图和附表清单
% \listoffigures           % 插图清单
% \listoftables            % 附表清单
% % \listoffiguresandtables  % 插图和附表清单

% 符号对照表
% !TeX root = ../thuthesis-yyy.tex

% \begin{denotation}[3cm]
%   \item[PI] 聚酰亚胺
%   \item[MPI] 聚酰亚胺模型化合物,N-苯基邻苯酰亚胺
%   \item[PBI] 聚苯并咪唑
%   \item[MPBI] 聚苯并咪唑模型化合物,N-苯基苯并咪唑
%   \item[PY] 聚吡咙
%   \item[PMDA-BDA] 均苯四酸二酐与联苯四胺合成的聚吡咙薄膜
%   \item[MPY] 聚吡咙模型化合物
%   \item[As-PPT] 聚苯基不对称三嗪
%   \item[MAsPPT] 聚苯基不对称三嗪单模型化合物,3,5,6-三苯基-1,2,4-三嗪
%   \item[DMAsPPT] 聚苯基不对称三嗪双模型化合物(水解实验模型化合物)
%   \item[S-PPT] 聚苯基对称三嗪
%   \item[MSPPT] 聚苯基对称三嗪模型化合物,2,4,6-三苯基-1,3,5-三嗪
%   \item[PPQ] 聚苯基喹噁啉
%   \item[MPPQ] 聚苯基喹噁啉模型化合物,3,4-二苯基苯并二嗪
%   \item[HMPI] 聚酰亚胺模型化合物的质子化产物
%   \item[HMPY] 聚吡咙模型化合物的质子化产物
%   \item[HMPBI] 聚苯并咪唑模型化合物的质子化产物
%   \item[HMAsPPT] 聚苯基不对称三嗪模型化合物的质子化产物
%   \item[HMSPPT] 聚苯基对称三嗪模型化合物的质子化产物
%   \item[HMPPQ] 聚苯基喹噁啉模型化合物的质子化产物
%   \item[PDT] 热分解温度
%   \item[HPLC] 高效液相色谱(High Performance Liquid Chromatography)
%   \item[HPCE] 高效毛细管电泳色谱(High Performance Capillary lectrophoresis)
%   \item[LC-MS] 液相色谱-质谱联用(Liquid chromatography-Mass Spectrum)
%   \item[TIC] 总离子浓度(Total Ion Content)
%   \item[\textit{ab initio}] 基于第一原理的量子化学计算方法,常称从头算法
%   \item[DFT] 密度泛函理论(Density Functional Theory)
%   \item[$E_a$] 化学反应的活化能(Activation Energy)
%   \item[ZPE] 零点振动能(Zero Vibration Energy)
%   \item[PES] 势能面(Potential Energy Surface)
%   \item[TS] 过渡态(Transition State)
%   \item[TST] 过渡态理论(Transition State Theory)
%   \item[$\increment G^\neq$] 活化自由能(Activation Free Energy)
%   \item[$\kappa$] 传输系数(Transmission Coefficient)
%   \item[IRC] 内禀反应坐标(Intrinsic Reaction Coordinates)
%   \item[$\nu_i$] 虚频(Imaginary Frequency)
%   \item[ONIOM] 分层算法(Our own N-layered Integrated molecular Orbital and molecular Mechanics)
%   \item[SCF] 自洽场(Self-Consistent Field)
%   \item[SCRF] 自洽反应场(Self-Consistent Reaction Field)
% \end{denotation}



% 也可以使用 nomencl 宏包,需要在导言区
% \usepackage{nomencl}
% \makenomenclature

% 在这里输出符号说明
% \printnomenclature[3cm]

% 在正文中的任意为都可以标题
% \nomenclature{PI}{聚酰亚胺}
% \nomenclature{MPI}{聚酰亚胺模型化合物,N-苯基邻苯酰亚胺}
% \nomenclature{PBI}{聚苯并咪唑}
% \nomenclature{MPBI}{聚苯并咪唑模型化合物,N-苯基苯并咪唑}
% \nomenclature{PY}{聚吡咙}
% \nomenclature{PMDA-BDA}{均苯四酸二酐与联苯四胺合成的聚吡咙薄膜}
% \nomenclature{MPY}{聚吡咙模型化合物}
% \nomenclature{As-PPT}{聚苯基不对称三嗪}
% \nomenclature{MAsPPT}{聚苯基不对称三嗪单模型化合物,3,5,6-三苯基-1,2,4-三嗪}
% \nomenclature{DMAsPPT}{聚苯基不对称三嗪双模型化合物(水解实验模型化合物)}
% \nomenclature{S-PPT}{聚苯基对称三嗪}
% \nomenclature{MSPPT}{聚苯基对称三嗪模型化合物,2,4,6-三苯基-1,3,5-三嗪}
% \nomenclature{PPQ}{聚苯基喹噁啉}
% \nomenclature{MPPQ}{聚苯基喹噁啉模型化合物,3,4-二苯基苯并二嗪}
% \nomenclature{HMPI}{聚酰亚胺模型化合物的质子化产物}
% \nomenclature{HMPY}{聚吡咙模型化合物的质子化产物}
% \nomenclature{HMPBI}{聚苯并咪唑模型化合物的质子化产物}
% \nomenclature{HMAsPPT}{聚苯基不对称三嗪模型化合物的质子化产物}
% \nomenclature{HMSPPT}{聚苯基对称三嗪模型化合物的质子化产物}
% \nomenclature{HMPPQ}{聚苯基喹噁啉模型化合物的质子化产物}
% \nomenclature{PDT}{热分解温度}
% \nomenclature{HPLC}{高效液相色谱(High Performance Liquid Chromatography)}
% \nomenclature{HPCE}{高效毛细管电泳色谱(High Performance Capillary lectrophoresis)}
% \nomenclature{LC-MS}{液相色谱-质谱联用(Liquid chromatography-Mass Spectrum)}
% \nomenclature{TIC}{总离子浓度(Total Ion Content)}
% \nomenclature{\textit{ab initio}}{基于第一原理的量子化学计算方法,常称从头算法}
% \nomenclature{DFT}{密度泛函理论(Density Functional Theory)}
% \nomenclature{$E_a$}{化学反应的活化能(Activation Energy)}
% \nomenclature{ZPE}{零点振动能(Zero Vibration Energy)}
% \nomenclature{PES}{势能面(Potential Energy Surface)}
% \nomenclature{TS}{过渡态(Transition State)}
% \nomenclature{TST}{过渡态理论(Transition State Theory)}
% \nomenclature{$\increment G^\neq$}{活化自由能(Activation Free Energy)}
% \nomenclature{$\kappa$}{传输系数(Transmission Coefficient)}
% \nomenclature{IRC}{内禀反应坐标(Intrinsic Reaction Coordinates)}
% \nomenclature{$\nu_i$}{虚频(Imaginary Frequency)}
% \nomenclature{ONIOM}{分层算法(Our own N-layered Integrated molecular Orbital and molecular Mechanics)}
% \nomenclature{SCF}{自洽场(Self-Consistent Field)}
% \nomenclature{SCRF}{自洽反应场(Self-Consistent Reaction Field)}



% 正文部分
\mainmatter
% !TeX root = ../thuthesis-yyy.tex

\chapter{引言}


\section{课题背景}

(课题背景与目标。)

操作系统(Operating System,OS)是计算机系统的核心软件,承担着管理和协调计算机硬件与软件资源的重任,其性能和功能直接影响着整个系统的运行效率和用户体验。而以Linux为代表的宏内核操作系统凭借其在资源管理和系统调用方面的高效性,在服务器、个人计算机等领域被广泛应用,支撑着大量业务的稳定运行与高效管理。但随着信息技术领域的快速发展,云计算、物联网等新兴技术不断涌现,计算机设备的应用场景日益复杂,对操作系统的灵活性和可扩展性提出了更高的要求。面对种类繁多、功能各异的设备,操作系统需要灵活适配、快速响应不同设备的多样化需求,而传统的宏内核操作系统将所有功能均放在单一的内核空间中实现,服务高度集中化,虽然能避开复杂的进程间通信机制,减少用户态和内核态之间的切换开销,提高系统调用效率,却也导致了内核代码规模庞大,可维护性和可扩展性较差,难以复用到资源差异较大的异构平台,而未熟练掌握内核开发技能的用户又难以针对不同场景需求开发适配的内核,因此一个能支持用户自由搭建所需功能的组件化宏内核操作系统便显得尤为重要(BrickOS: 面向异构硬件资源的积木式内核)。

任务管理作为宏内核操作系统的核心功能,负责高效调度任务、合理分配资源等关键工作,从而保障整个系统的有序运行,因此在搭建组件化操作系统的过程中首先且必须要实现的便是任务管理组件。(补充)

本课题旨在基于支持组件可重用设计的组件化操作系统基层架构ArceOS,搭建其上层系统调用,从而设计并封装出能直接支持Linux应用的组件化宏内核操作系统的任务管理组件,帮助定制更便捷、高效、安全的宏内核操作系统。



\section{相关研究工作}

(
1. 组件化操作系统的理论研究现状;

2. 对已有的OS实现现状(如byteos、dragonos、新绽)进行简单分析与介绍。
)

在组件化操作系统的理论研究方面,众多学者均对其架构方式、设计原则以及运行机制进行了深入的探讨。(待补充 综述)

目前,已有多个操作系统对宏内核操作系统进行了各自的设计尝试与实现,其中ByteOS、DragonOS以及星绽操作系统均使用Rust语言开发并提供了完备的宏内核相关功能,兼容Linux应用,他们的实现为本课题的设计研究提供了宝贵的借鉴经验。

ByteOS是一个开源的操作系统项目,将进程管理、内存管理、文件系统等核心模块紧密集成,实现了较为简洁的宏内核架构。在任务管理方面,ByteOS使用进程控制块(PCB)作为管理进程的数据结构,记录进程状态、优先级、程序计数器等关键信息,线程管理同样使用类似的线程控制块(TCB)来管理线程的上下文信息,包括陷阱帧、信号掩码、子线程退出标志、信号队列、退出信号和线程退出码等。ByteOS支持多种硬件平台,如riscv64、aarch64、x86\_64、loongarch64等,并能在这些平台上运行Linux应用程序。

DragonOS同样基于宏内核架构,是一个面向云计算轻量化场景的,完全自主内核的64位操作系统。其内核包含完备的宏内核操作系统所需功能,并实现了优秀的内存管理模块,对内核空间和用户空间的内存分配、释放、管理等进行了封装,在任务管理方面也实现了完善的进程管理和多核调度,并支持内核线程和kthread机制,为内核线程的创建和管理提供了方便的接口,保证系统能高效处理各种任务。

星绽操作系统则致力于充分发挥Rust潜力, 以安全高效的方式实现Linux ABI并满足专有内核模块的业务需求。星绽操作系统的任务管理机制在任务创建、调度、同步与通信以及退出等方面都有其独特的设计和实现,在任务调度上也会根据任务的状态和资源需求进行动态调整。



\section{组件化操作系统}

(
1. 结合上面的研究现状,介绍什么是宏内核下的组件化操作系统;

2. 介绍任务管理模块:进程与线程、任务状态与生命周期、任务调度策略;

3. 简要描述宏内核架构下的任务管理地位,并说明任务管理是怎样参与到一个操作系统宏内核的构建与运行中去的。
)

宏内核下的组件化操作系统在传统宏内核基础架构上进行创新发展,将操作系统的各个核心功能封装为独立的组件,各功能不再紧耦合,而是在通过组件间的标准化接口进行交互与协作。这种架构模式既能保留宏内核的高性能优势,又能赋予系统良好的可维护性与灵活的可扩展性,从而大幅拓展宏内核操作系统的应用范围。

任务管理模块作为组件化操作系统的关键组成部分,通过管理进程与线程、控制任务状态与生命周期以及灵活进行任务调度来实现对宏内核核心功能的支持。进程作为系统资源分配和CPU调度的基本单位,是程序在计算机中的一次动态执行过程,每个进程都拥有独立的地址空间及系统资源,一般通过进程控制块来实现对进程状态、优先级、资源占用等信息的精确记录。而线程作为进程内的轻量级执行单元,则通过线程控制块来实现,通过共享进程的资源实现协同工作,但也拥有独立的线程栈和程序计数器以实现独立调度,保证资源的高效利用。任务状态描述了任务在执行过程中的不同阶段,常见的任务状态包括就绪态、运行态、阻塞态和终止态,任务状态的转换构成了任务的生命周期,通过任务管理模块进行监控和控制。而任务调度策略的核心目标是平衡公平性与效率,基于此存在一系列策略对任务的状态进行管理与切换,常见策略包括时间片轮转(Round-Robin,RR)、先来先服务(First Come First Serve,FCFS)、多级反馈队列调度(Multi-Level Feedback Queue, MLFQ)等,通过灵活运用各种调度策略,操作系统能根据任务特点和资源情况对CPU资源进行高效分配,提升系统效率,满足不同应用场景的需求。

在宏内核架构中,任务管理模块贯穿操作系统的整个运行过程,在系统启动阶段负责任务相关数据结构与资源的初始化,在系统运行过程中则与内存管理、文件系统等模块协作,调度任务、管理资源,并在系统运行结束后回收相关资源,保障系统能够高效、稳定地运行。


\section{本文工作}

(概述本文主要工作,如何基于单位内核组件实现直接支持Linux应用的组件化宏内核。)

本文旨在实现高效、完备、可靠的宏内核任务管理组件,将基于基座代码ArceOS中的单位内核组件以及Starry-Next整体框架展开深入研究与实践,通过对任务管理相关数据结构的设计运用、系统调用接口的补全实现以及对整个任务管理组件功能与性能的集成测试,实现对组件化宏内核的核心支持。

此外,本课题同时为ArceOS及Starry-Next框架中的任务相关功能与接口提供了详细的说明文档,以供使用者参阅。

本文分为6个章节,在第一章介绍课题背景及组件化宏内核相关研究工作,在第二章将介绍本课题所使用的基座代码ArceOS的整体架构以及任务管理相关单元模块,在第三章则将对Starry-Next框架进行介绍、接口分析及任务管理模块的相关功能,在第四章将介绍任务管理组件的设计与实现,在第五、六章则将分别对组件进行测试分析以及最终总结。
% !TeX root = ../thuthesis-yyy.tex

\chapter{基座代码架构分析}

\section{整体架构概述}

介绍ArceOS,给出总体分析。



\section{基座代码的任务管理组件}

对 ArceOS 任务管理相关核心组件进行介绍,功能介绍与实现分析:

1. axruntime:运行时;

2. axtask:任务管理与调度;

3. axns:命名空间。



\section{实现基础宏内核操作系统}

如何使用基座框架实现基础宏内核操作系统并运行简单用户程序,这一部分之前总结过,需将其整理拟合成符合论文要求的表达。

% !TeX root = ../thuthesis-yyy.tex

\chapter{Starry-Next架构分析}

\section{整体架构概述}

简要介绍即可



\section{基座代码的任务管理组件}

对 ArceOS 任务管理相关核心组件进行介绍,功能介绍与实现分析:

1. axruntime:运行时;

2. axtask:任务管理与调度;

3. axns:命名空间。



\section{实现基础宏内核操作系统}

如何使用框架实现基础宏内核操作系统并运行简单用户程序,这一部分之前总结过,需将其整理拟合成符合论文要求的表达。

% !TeX root = ../thuthesis-yyy.tex

\chapter{任务管理组件设计与实现}

\section{开发环境与工具}

编程语言、开发工具、实验环境的搭建。



\section{总体设计架构}

描述任务管理组件这一个整体如何运行:

1. 功能划分;

2. 与其他组件的交互关系;

3. 任务数据结构设计;

4. 任务创建、销毁、切换设计。



\section{设计与完成思路}

如何一步步实现相关系统调用:

1. 重要系统调用的介绍与分析;

2. 系统调用逐步实现安排;

3. 如何基于arceOS基座代码进行系统调用的实现与接入;

4. 不同(任务组件内以及不同组件间)系统调用如何相互影响和依赖。



\section{实现细节}

1. 如何在实际Starry-Next框架中实现相关系统调用;

2. 结合进展文档给出实现过程中的问题、挑战与解决方法。

% !TeX root = ../thuthesis-yyy.tex

\chapter{任务管理组件测试与分析}

\section{测试用例}

使用操作系统大赛测例作为测试用例与指标:

1. 介绍评价指标(测例);

2. 介绍测试方式与主要测试内容。


\section{自己编写简单测例}

如何自己编写简单测例:过程与例子。



\section{测试结果分析}

测试过程与结果。

% !TeX root = ../thuthesis-yyy.tex

\chapter{结论}

% \section{开发环境与工具}

总结与展望。


% 参考文献
\bibliography{ref/refs}  % 参考文献使用 BibTeX 编译
% \printbibliography       % 参考文献使用 BibLaTeX 编译

% 附录
\appendix
% \input{data/appendix-survey}       % 本科生:外文资料的调研阅读报告
% \input{data/appendix-translation}  % 本科生:外文资料的书面翻译
% !TeX root = ../thuthesis-yyy.tex

\chapter{补充内容}

附录是与论文内容密切相关、但编入正文又影响整篇论文编排的条理和逻辑性的资料,例如某些重要的数据表格、计算程序、统计表等,是论文主体的补充内容,可根据需要设置。

附录中的图、表、数学表达式、参考文献等另行编序号,与正文分开,一律用阿拉伯数字编码,
但在数码前冠以附录的序号,例如“图~\ref{fig:appendix-figure}”,
“表~\ref{tab:appendix-table}”,“式\eqref{eq:appendix-equation}”等。


\section{插图}

% 附录中的插图示例(图~\ref{fig:appendix-figure})。

\begin{figure}
  \centering
  \includegraphics[width=0.6\linewidth]{example-image-a.pdf}
  \caption{附录中的图片示例}
  \label{fig:appendix-figure}
\end{figure}


\section{表格}

% 附录中的表格示例(表~\ref{tab:appendix-table})。

\begin{table}
  \centering
  \caption{附录中的表格示例}
  \begin{tabular}{ll}
    \toprule
    文件名          & 描述                         \\
    \midrule
    thuthesis.dtx   & 模板的源文件,包括文档和注释 \\
    thuthesis.cls   & 模板文件                     \\
    thuthesis-*.bst & BibTeX 参考文献表样式文件    \\
    thuthesis-*.bbx & BibLaTeX 参考文献表样式文件  \\
    thuthesis-*.cbx & BibLaTeX 引用样式文件        \\
    \bottomrule
  \end{tabular}
  \label{tab:appendix-table}
\end{table}


\section{数学表达式}

% 附录中的数学表达式示例(式\eqref{eq:appendix-equation})。
\begin{equation}
  \frac{1}{2 \uppi \symup{i}} \int_\gamma f = \sum_{k=1}^m n(\gamma; a_k) \mathscr{R}(f; a_k)
  \label{eq:appendix-equation}
\end{equation}


\section{文献引用}

附录\cite{dupont1974bone}中的参考文献引用\cite{zhengkaiqing1987}示例
\cite{dupont1974bone,zhengkaiqing1987}。

\printbibliography


% 其他部分
\backmatter

% 致谢
% !TeX root = ../thuthesis-yyy.tex

\begin{acknowledgements}
  衷心感谢导师戴桂兰老师和陈渝副教授对本人的悉心指导,他们的言传身教和不懈的鼓励帮助将使我终生受益。

  感谢软件所以及清华东升基地全体老师和同学们的热情帮助和支持!
\end{acknowledgements}


% 声明
% 各类开题报告通常不需要
\statement[page-style=empty]  % 编译生成的声明页默认不含页眉页脚,以避免页码变化带来问题
% 在提交终稿时,插入签字后的扫描件 scan-statement.pdf,并添加页眉页脚
% \statement[page-style=plain, file=scan-statement.pdf]
% 如确实需要在电子版中直接页眉页脚,则使用
% \statement[page-style=plain]

% 本科生的综合论文训练记录表(扫描版)
% \record{file=scan-record.pdf}

\end{document}
